\documentclass{article}

\usepackage[utf8]{inputenc}
\usepackage[margin=1.0in]{geometry}
\usepackage{bussproofs}
\usepackage{listings}
\usepackage{xspace}
\usepackage{multirow}
\usepackage{setspace}

\usepackage{effigy-definitions}

\renewcommand{\baselinestretch}{1.5}

\title{Effigy}
\author{Henry Blanchette}
\date{October 2019}

\begin{document}

\maketitle

% -------------------------------------------------------------------------------------
% -------------------------------------------------------------------------------------
% -------------------------------------------------------------------------------------
\section{Conventions}

\begin{center} \begin{tabular}{l|l}
  Notation               & Meaning                        \\ \hline
  $e$                    & expression                     \\
  $x, y, z$              & term variable                  \\
  $v$                    & value                          \\
  $h$                    & handler                        \\
  $\tau, \sigma, \rho$   & type                           \\
  $\list{x}$             & list of $x$                    \\
  $\phi$                 & effect                         \\
  $\Phi$                 & set of effects                 \\
  $\handler{\Phi}{\tau}$ & $\Phi$-handler yielding $\tau$ \\
  $\effect{\Phi}{\tau}$  & $\Phi$-effect yielding $\tau$
 \end{tabular} \end{center}

% -------------------------------------------------------------------------------------
% -------------------------------------------------------------------------------------
% -------------------------------------------------------------------------------------
\newpage
\section{Grammar}

\[ \begin{array}{rcl}
  \tit{program}     & :=  & \list{\tit{declaration}} ~ \tsf{Main} ~ := ~ e ~ .
  \\
  \tit{declaration} & ::= &
  \tsf{Effect} ~ \phi:\tau. \alt
  \tsf{Value} ~ x : \tau := e.
  \\
  e                 & ::= &
  x \in \tit{Term-Names} \alt
  v \alt
  e ~ e \alt
  \{ ~ \list{e ~ ;} ~ e ~ \} \alt
  \tsf{do} ~ e ~ \tsf{with} ~ e
  \\
  v                 & :=  &
  v \in \tit{Primitive-Values} \alt
  x \mt e \alt
  h
  \\
  h                 & ::= &
  \tsf{handler} ~ \{ ~
  \tsf{effect} ~ (x ~ x) ~ x \mt e ~ | ~
  \tsf{value} ~ x \mt e ~ | ~
  \tsf{run} ~ x \mt e
  ~ \}
  \\
  \tau              & ::= &
  \tau \in \tit{Types} \alt
  \tau \to \tau \alt
  \handler{\Phi}{\tau} \alt
  \effect{\Phi}{\tau}
  \\
  \phi              & \in & \tit{Effect-Names}
  \\
  \Phi              & \in & \mathcal{P}(\tit{Effect-Names})
 \end{array} \]

% -------------------------------------------------------------------------------------
% -------------------------------------------------------------------------------------
% -------------------------------------------------------------------------------------
\section{Definitions}

% -------------------------------------------------------------------------------------
% -------------------------------------------------------------------------------------
\subsection{Lifting to Effects}

Since $\effect{\Phi}{\tau}$ is the type of terms that have effects $\Phi$ and result in $\tau$,
if $\Phi = \varnothing$ then this $\effect{\Phi}{\tau}$ is the type of terms that have
no effects and result in $\tau$ i.e. just $\tau$. The following bijection reflects this.
\begin{align*}
 \tsf{lift} : \tau & \iff \effect{\varnothing}{\tau} \tag{lift to effect}
\end{align*}

% -------------------------------------------------------------------------------------
% -------------------------------------------------------------------------------------
% \subsection{Effect Monad}
% TODO

% -------------------------------------------------------------------------------------
% -------------------------------------------------------------------------------------
\subsection{Partial Ordering for Effects}

The power set of effects, $\mathcal{P}(\tit{Effects})$,
has a partial order induced by set inclusion.
Define meet, $\meet$, to be set intersection
and join, $\join$, to be set union.
Note that $\varnothing$ is the minimal element and $\tit{Effects}$ is the maximal element.

$\meet$ and $\join$ is overloaded to apply to types as well,
via the following definitions:
\[ \begin{array}{lclcl}
  \effect{\Phi}{\tau} & \meet & \effect{\Phi'}{\tau'}
                      & :=    & \effect{(\Phi \meet \Phi')}{\tau'}
  \\
  \tau                & \meet & \tau'
                      & :=    & \tsf{lift} \tau \meet \tsf{lift} \tau'
  \\[1em]
  \effect{\Phi}{\tau} & \join & \effect{\Phi'}{\tau'}
                      & :=    & \effect{(\Phi \join \Phi')}{\tau'}
  \\
  \tau                & \join & \tau'
                      & :=    & \tsf{lift} \tau \join \tsf{lift} \tau'
 \end{array} \]

% -------------------------------------------------------------------------------------
% -------------------------------------------------------------------------------------
% -------------------------------------------------------------------------------------
\newpage
\section{Typing}

\subsection{Functional Foundations}
This first set of typing rules reflect a traditional functional-style type system.

\[ \begin{array}{ccc}
  % judgement introduction
  \inference
  { x:\tau \in \context }
  { \Gamma \entails x:\tau }

   &

  % lambda introduction
  \inference
  { \context, x:\sigma \entails e:\tau }
  {\Gamma \entails x \mt e : \sigma \to \tau }

   &

  % lambda destruction
  \inference
  { \context \entails e_1 : \sigma \to \tau \sep
   \context \entails e_2 : \sigma }
  { \context \entails e_1 ~ e_2 : \tau }

  \\\\
 \end{array} \]

\noindent
\subsection{Sequences of effects}
This next set of typing rules allow for the introduction of sequenced terms.
In the usual pure context, such expressions are unneeded because the sequencing of terms
must have no effect on computation.
However, with the introduction of effects in mind, sequencing allows for ``complex'' effects
that perform multiple effects in some order, possibly using the results of earlier
effectual computations later in the sequence.

\[ \begin{array}{ccc}

  % sequence introduction
  \inference
  {
   \context \entails e:\tau
  }{
   \context \entails \{ ~ e ~ \} : \tau
  }

   &

  \inference
  {
   \context \entails e_1:\sigma \sep
   \context \entails e_2:\tau
  }{
   \context \entails \{ ~ e_1 ~ ; ~ e_2 ~ \} : \sigma \join \tau
  }

   &

  \inference
  {
   \context \entails \{ ~ e_1 ~ ; ~ \cdots ~ ; ~ e_n ~ \} : \sigma \sep
   \context \entails e_{n+1} : \tau
  }{
   \context \entails \{ ~ e_1 ~ ; ~ \cdots ~ ; ~ e_n ~ ; ~ e_{n+1} ~ \}
   : \sigma \join \tau
  }

  \\\\
 \end{array} \]

\noindent
\subsection{Handlers for effects}
The next rule defines the type for each component of a handler.
The rule following it defines the ``application'' of handlers on
effectual computations in order to yield a pure(er) result.


\[ \begin{array}{cc}

  % handler introduction
  \inference
  {
   \begin{array}{c}
    \phi : \sigma \to \effect{\set{\phi}}{\tau}
    \\
    \context, x:\sigma, k:\tau \to \effect{\set\phi}{\rho}
    \entails e_\phi:\sigma \to \effect{\set\phi}{\rho}
   \end{array}
   \sep
   \begin{array}{c}
    \context, x:\sigma \entails e_v:\sigma \to \effect{\set\phi}{\rho}
    \\
    \context, g:\sigma \to \effect{\set\phi}{\rho} \entails e_r:\effect{\set\phi}{\rho}
   \end{array}
  }{
   \context \entails
   \tsf{handler} ~ \{ ~
   \tsf{effect} ~ (\phi ~ x) ~ k \mt e_\phi ~ | ~
   \tsf{value} ~ x \mt e_v ~ | ~
   \tsf{run} ~ g \mt e_r
   ~ \} :
   \handler{\set{\phi}}{\rho}
  }

  \\\\

  \inference
  {
   \context \entails e:\effect{\Phi}{\rho}
   \sep
   \context \entails h:\handler{\Phi'}{\rho}
  }{
   \context \entails
   \tsf{do} ~ e ~ \tsf{with} ~ h : \effect{(\Phi \setminus \Phi')}{\rho}
  }
 \end{array} \]

\newpage
\section{Rewriting}

\[ \begin{array}{llcl}
    \tit{application} &
    (x \mt e) ~ v
    &\rwto&
    [v/x] e
    \\

    \tit{handle effect} &
    \tsf{do}~\{~\phi~e_1~;~e_2~\}~\tsf{with} && \\ &
    \tsf{handler}~\{~\tsf{effect}~(\phi~x)~k \mt e_\phi~\}
    &\rwto&
    TODO
    \\

    \tit{handle value} &
    \tsf{do}~\{~e_1~;~\cdots~;~e_{n-1}~;~e_n~\}~\tsf{with} && \\ &
    \tsf{handler} ~ \{~\tsf{value}~x \mt e_v~\}
    &\rwto&
    (\tsf{do}~\{~e_1~;~\cdots~;~e_{n-1}~\}~\tsf{with} \\ &&&
    \tsf{handler} ~ \{~\tsf{value}~x \mt e_v~\})
    \\

    \tit{handle run} &
    \tsf{do}~\{~e~\}~\tsf{with} && \\ &
    \tsf{handler} ~ \{~\tsf{run}~g \mt e_r~\}
    &\rwto&
    e_r
    \\

\end{array} \]

\end{document}
