\chapter{Freer Monadic Effects}
\label{chapter:freer-monadic-effects}

\section{Outline}
\begin{enumerate}
  \item Consideration of the problem of \tit{composing monads}; no systems so far have facilitated this while maintaining type-checking
  \item Definition and context for freer monads in category theory and computer science, in relation to monads as previously discussed in chapter~\ref{chapter:monadic-effects}
  \item Explanation of how freer monads can model effects
  \begin{enumerate}
    \item left Kan extension (Lan)
    \item act as a sort of monadic implementation of algebraic effect handlers
  \end{enumerate}
  \item Demonstration of stateful freer monad, paralleling monadic example from chapter~\ref{chapter:monadic-effects}
  \item Discussion of advantages and disadvantages of freer monadic effects
  \begin{enumerate}
    \item fully-typed system for algebraic effect handlers; all the advantages of algebraic effect handlers
    \item facilitates generally composable effects
    \item not implementable in non-dependently typed languages like Haskell, OCaml, SML, etc. since it requires extensive type-level manipulation
    \item potentially very user-unfriendly, incurring many of the original problems with monadic effects
  \end{enumerate}
\end{enumerate}
