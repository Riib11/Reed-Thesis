\chapter{Appendix: Conventions}
\label{app:conventions}

\section{Languages}

The discussion of formal languages requires the use of several different languages simultaneously.
The following is a list of the languages used in this work:
\begin{itemize}
  \item \tbf{narrative (informal) language}:
    The top-level language used to narrate this work in an intuitive prose;
    English.
  \item \tbf{logical meta-language}:
    Formalized expressions that make no extra-logical assumptions.
    Contains expressions such as ``The proposition $A$ implies the proposition $A$.''
  \item \tbf{mathematical meta-language}:
    Mathematical expressions of generalized programs, where terms range more freely than in actual valid programs in the programming language.
    Relies on a mathematical context not explicit in this work.
    Contains expressions such as ``$f : α \to β \to γ$.''
  \item \tbf{programming language}:
    Programs that are fully defined by the contents of this work, relying on no external context.
    Necessarily conforms to the given syntax, and its expressions have no meaning beyond the given rules that apply to them.
    Contains expressions such as ``\code|term id (α:Type) (a:α) : α ≔ a.|.''
\end{itemize}

\section{Fonts}

The use of many different languages simultaneously, as described by the previous section, has the unfortunate consequence of certain expressions seeming ambiguous to the reader.
In order to mitigate this problem, each language and certain kinds of phrases are designated a font.
The following fonts are designated in this way:
\begin{itemize}
  \item \tbf{normal font}: narrative language. E.g. ``the usual.''
  \item \tbf{italic font}: emphasis in narrative language, especially of new and important words. E.g. ``\ep{emphasize this}.''
  \item \tbf{bold font}: indicating the definition of new terms keywords in narrative language. E.g. ``\kw{important definition}.''
  \item \tbf{small-caps font}: names in logical meta-language. E.g. ``\rulename{The-Golden-Rule}.''
  \item \tbf{sans-serif font}: mathematical meta-language. E.g. ``$f(x) = (f \circ f)(x)$.''
  \item \tbf{monospace font}: programming language. E.g. ``\code|this is some code|.''
\end{itemize}

\section{Names}

Naming conventions in programs:
\begin{itemize}
  \item \tbf{terms}: lower-case english word/phrase in kabob-case. E.g. \code|this-is-a-term|, \code|this-is-not-a-term|, \code|map|.
  \item \tbf{term variables}: lower-case english letter. E.g. \code|a| \code|b| \code|c|.
  \item \tbf{types}: lower-case english word/phrase. E.g. \code|list|, \code|optional|, \code|unit|.
  \item \tbf{type-classes}: capitalized english word/phrase. E.g. \code|Monad|, \code|Animal|, \code|Functor|.
  \item \tbf{type variables}: lower-case greek letter. E.g. \code|α|, \code|β|, \code|φ|, \code|υ|.
  \item \tbf{type variables of higher order}: capital english letter. E.g \code|M|, \code|A|, \code|L|.
\end{itemize}
