\chapter{Pure Computation}
\label{ch:introduction}

% ------------------------------------------------------------------------------
% ------------------------------------------------------------------------------
% ------------------------------------------------------------------------------
\section{Introduction}

% Programming languages are a formal way of expressing computation.

In 1936, Alan Turing proposed \ep{Turing machines}%
\footnote{
  Turing first described Turing machines, but did not name them such, in \cite{Turing1936}.
}
as a formal representation of \ep{computation}.
A Turing machine performs computation by taking as input a string of symbols (a \ep{program}), reading the symbols, changing its own internal state via some pre-defined rules prescribed by its construction, and writing an output.
A Turing machine's \ep{program} amounts to a series of instructions dictating the order and specifics of these activities (e.g. what to write, what to read, which state to change to).
The language style resulting from using this model as a programmer's interface to computation will be detailed in chapter~\ref{ch:simple-effects} as \ep{imperative programming}.

Also in 1936, Alonzo Church proposed the \ep{\lc}%
\footnote{
  The \ep{\lc} was first used in \cite{Church1936}.
  Importantly for this work however, the \ep{simply-typed \lc} was introduced in 1940 by Church in \cite{Church1940}.
}
as a formal representation of \ep{computation}.
In contrast to Turing machines, the \lc exists purely as a language of formulae with no interactive, stateful machine attached.
The \lc represents computation as the process of transforming \lc-terms into other \lc-terms according to \ep{rewrite}, or \ep{reduction}, rules.
A \lc \ep{program} is a term that, when provided the proper inputs, reduces to the function's output.
In this way, \lc programs look much more like mathematical functions than computer code.
The language style resulting from using this model as a programmer's interface to computation will be detailed in chapter~\ref{ch:simple-effects} as \ep{declarative programming}.

Both of these highly abstract systems are still used prolifically in the theoretical computer science community, but how well do they map on to how real-world programmers treat computers?%
\footnote{
  I do not mean to claim that the original purpose of the systems was to map onto future programming languages.
  However, the comparison is useful to see what is commonly used in real-world programming but missing in the theoretical framework.
}
On the one hand, Turing machines at a high level behave very similarly to programmers' interface to computers, though the lowest-level activities dictated by a real-world computer are more complicatedly expressed.
\clearpage .

On the other hand, the \lc is almost completely%
\footnote{
  Though, importantly, there are a few programming language communities making use of \lc inspired languages such as Haskell, OCaml, and of course the legendary Lisp language.
}
removed from any reasonable description of real-world programs.
Only very small sections, such as simple mathematical operations, could be feasibly translated into \lc terms.
Why have real-world programming languages so heavily adopted a Turing-machine-like sequential-instruction model and not a \lc-like formulaic model for representing computation?

% One reason is that real-world programming languages usually express programs in terms of a controlled sequence of instructions rather than static mathematical definitions.
Though the simply-typed \lc has been proven to be exactly as \ep{expressive} as Turing machines (that is, there is a correspondence between pairs of Turing machines with their programs and \lc terms),
the \lc requires the entire behavior of each term to be specified within the term itself,
whereas a Turing machine can make use of its state to keep track of external contexts when reading each individual instruction of a program.
Real-world languages for real-world software heavily rely on externally-defined behavior, and so the Turing machine model better fits this need.
This observation is the inspiration for the concept of \ep{effects}, which will be detailed in chapter~\ref{ch:simple-effects}.

However, the \lc model still has significant benefits that are desirable yet lacking in many real-world programming languages.
\lc is a very \kw{safe} language --- that is, its programs are very easy to formally analyze and predict the behavior of.
This is in contrast to the \kw{dangerous} inescapable complexity of implicit dependencies on implicit contexts and externally-defined behavior of real-world languages, which is very difficult to formally analyze, test, and predict the behavior of.
This safety, though, is easily lost when the features needed for real-world programs --- namely, effects --- are added to a \lc-like language (as will be demonstrated by \LangB).
The goal of this thesis is to consider an alternative to this predicament by introducing a general way of implementing such features in a \lc-like language while still preserving safety.
The rest of this chapter introduces \LangA, a slightly-embellished version of the typed \lc that will serve as a basis for the later extensions.

Chapter~\ref{ch:simple-effects} expands the concept of and programming-language relevance of the concept of \ep{effects} (i.e. the features missing yet desired for real-world programming languages).
It then introduces \LangB (extension of \LangA), which takes a simple approach in implementing \ep{imperative} effects, using a selection of example effects: mutability, exception, nondeterminism, and input/output.
It ends with some considerations for imperative-effect framework.
Pros: it is good for quick and dirty programs that correspond closely to imperative languages' (in fact, the real-world language OCaml uses this effect framework).
Cons: it \ep{dangerously} lacks \ep{explicitly-typed} effects, programmable \ep{internal} effects, and a generalized language-structure for defining effects.

Chapter~\ref{ch:ME} introduces the concept of \ep{monad} (and its corresponding type-class) as a way to represent effects within \LangA.
It then introduces \LangC (a minimal extension of \LangA), which constructs a monad instance for each effect.
It ends with some considerations for the monadic-effect framework:
Pros: it is very \ep{safe} (esp. \ep{type-safe}; has explicitly-typed effects), promotes programmable internal effects, and does not rely on any reduction contexts.
Cons: it requires a lot of complicated type considerations from the programmer and does not allow the \ep{composition} of effects.

Chapter~\ref{ch:AEH} introduces the concept of \ep{algebraic effects with handlers} as an alternative way (to monads) to extend \LangA with effects while still preserving some safety.
It then introduces \LangD (major extension of \LangA), which provides a collection of language structures and reduction rules for hosting algebraic effects and handlers.
It ends with some considerations for the algebraic effect and handler framework.
Pros: It preserves some of the safety of \LangA without becoming as type-burdensome as \LangC, allows for the composition of effects (at the \ep{performance} level), and allows the definition of multiple ways to \ep{handle} the same effect performances.
Cons: it loses some important type-safety managed by \LangC and relies heavily on implicit reduction contexts.

Finally, chapter~\ref{ch:FME} introduces the concept of \ep{freer monads} as an implementation of algebraic effects and handlers into \LangC as a monad.
It then introduces \LangE (minimal extension of \LangC), which constructs a single freer monad instance as a general monad for all effects.
It demonstrates how \LangE can be used, with a \ep{stacked-freer monad}, to allow the composability of freer-monadic effects in a way mimicking algebraic effects and handlers.
It ends with some considerations for the freer-monadic effect framework.
Pros: it combines the pros of monadic effects and algebraic effects with handlers, while still maintaining the type-safety of \LangA.
Cons: it still requires some complicated type considerations in the background (although they need not be too explicitly exposed to the programmer in order to typically use them).

% ------------------------------------------------------------------------------
% ------------------------------------------------------------------------------
% ------------------------------------------------------------------------------
\newpage
\section{Language \LangA}
\label{sec:LangA}

This section presents the language \LangA, a slightly-embellished version of the typed \lc.
There are three parts to working with \LangA programs: \ep{syntax}, \ep{typing}, and \ep{reduction}.
Syntax defines a grammar for what kinds of expression are \kw{well-formed} i.e. are meaningful strings of symbols.
Typing defines an inference scheme for which kinds of well-formed are \kw{well-typed} i.e. strings that correspond to valid programs.
Finally, reduction defines the logical steps used for \kw{evaluating} well-typed terms i.e. computing.

Keep in mind that \LangA and all other languages defined in this thesis are \ep{not} sensitive to the white-space (i.e. spaces and line-returns), which is manipulated freely for the sake of readability.
For example, the string
\begin{snippet}
a (b c)
  d
  (e
    f)
\end{snippet}
is read as equivalent to \code|a (b c) d (e f)|.
This is not the norm in many programming languages (e.g. Python).

\newpage

% ------------------------------------------------------------------------------
\subsection{Syntax}

% give a sort of narrative introduction to programming

Table~\ref{tab:LangA-syntax} fully defines the syntax \LangA.
The rest of the section explains the significance of and how to interpret \LangA programs.

\begin{table}[H]
\label{tab:LangD-syntax}
\centering
\captionof{table}{Syntax for \LangD}
\begin{tabular}{r|L{25em}|l}
  \kw{metavariable} & \kw{generator} & \kw{name}
\\ \hline \hline && \\
  \mvar{declaration}
    & \code`resource $\mvar{resource-name}$ $〚$ ($\mvar{type-param}$:$\mvar{kind}$) $〛$`
    & resource \\
    % & \code`           $〚$ ($\mvar{type-param}$:$\mvar{kind}$) $〛$` \\
    & \code`  { $〚$ $\mvar{action-name}$ $〚$ ($\mvar{type-param}$:$\mvar{kind}$) $〛$`
    & definition \\
    & \code`        : $\mvar{type}$ ↗ $\mvar{type}$ ; $〛$ }.`
  \\[1em]
    & \code`channel $\mvar{channel-name}$ : $\mvar{type}$.`
    & channel \\&& instantiation
\\[1em] \hline && \\
  \mvar{kind}
    & \code`Resource`
    & resource kind
\\[1em] \hline && \\
  \mvar{type}
    & \code`$\mvar{type}$ ↗ $\mvar{type}$`
    & action type
  \\[1em]
    & \code`$\mvar{type}$ ↘ $\mvar{type}$`
    & handler type
\\[1em] \hline && \\
  \mvar{term}
    & \code`$\mvar{channel-name}$#$\mvar{action-name}$ $\mvar{term}$`
    & performance
  \\[1em]
    & \code`handler $\mvar{term-name}$ $〚$ ($\mvar{type-param}$:$\mvar{kind}$) $〛$`
    & handler \\
    & \code`  : $\mvar{kind}$` \\
    & \code`  { $〚$ #$\mvar{action-name}$ $\mvar{term-parap}$ $\mvar{term-param}$` \\
    & \code`        ⇒ $\mvar{term}$ ; $〛$ }`
  \\[1em]
    & \code`$\mvar{term}$ with $\mvar{term}$`
    & handle \\&& performance
\\[1em]
\end{tabular}
\end{table}

% action
% perform
% do with

% resource $\mvar{resource-name}$ $〚$ ($\mvar{type-param}$:$\mvar{kind}$) $〛_*$ : Resource
%   { $\mvar{action-name}_1$ $〚$ ($\mvar{type-param}$:$\mvar{kind}$) $〛_1$ : $\mvar{type}_{1,\text{in}}$ ↗ $\mvar{type}_{1,\text{out}}$
%   ; $\cdots$
%   ; $\mvar{action-name}_n$ $〚$ ($\mvar{type-param}$:$\mvar{kind}$) $〛_n$ : $\mvar{type}_{n,\text{in}}$ ↗ $\mvar{type}_{n,\text{out}}$ }.


\newparagraph
Declarations are written at the top-level of a program, and declare globally-accessible \kw{names}.
Names can be declared either \ep{primitively} or \ep{constructively}.
A \kw{primitive} declaration is axiomatic in that it only declares the type of a new term name (or the kind of a new type name) without defining the term or type that the name is equal to.
For example, the type \code|unit| and its one term can be introduced primitively declared by the following.
%
\begin{snippet}
primitive type unit : kind.

primitive term • : unit.
\end{snippet}
%
A \kw{constructive} declaration does provide the term or type that a new term name or type name respectively is equal to.
In this way, the new name can expand into its definition wherever it is used.
For example, we can declare two synonyms for the previously declared names.
%
\begin{snippet}
type unit' : kind ≔ unit.

term •' : unit-again ≔ •.
\end{snippet}

There are three sorts of expressions: \ep{terms}, \ep{types}, and \ep{kinds}.
Firstly, \kw{terms} are the data of computation.
Such items as numbers, functions, strings, etc. expressible in \LangA are all terms.
There are three forms of terms: \ep{term atoms}, \ep{term functions}, and \ep{term applications}.
\kw{Term atoms} are simply names posited as terms.
For example: the number \code|1|, the boolean \code|true|, the string \code|"hello world"|, and the boolean function \code|not| are all term atoms.
\kw{Term functions} is a term that encodes a function of one parameter.
suppose we want to write a function equivalent to \code|not| ---
it takes one boolean parameter, $b$, then evaluates to \code|false| if $b$ is true and \code|true| if $b$ is true.
In \LangA this is written as follows.
%
\begin{snippet}
(b:boolean) ⇒ if b then false else true
\end{snippet}
%
Names on the left side of the ``\code| ⇒ |'' are the function's \ep{parameters} and the term on the left side is the function's \ep{body}.
We could also write the boolean and function of two parameters as follows.
%
\begin{snippet}
(b1:boolean) ⇒
  (b2:boolean) ⇒
    if b1
      then if b2 then true else false
      else false
\end{snippet}
%
Observe that this function of two parameters is really encoded as a function of one parameter, \code|b1|, to another function of one parameter, \code|b2|.%
\footnote{
  It is read this way because the infixed ``\code| => |'' is right-associative i.e. \code|a => b => c| implicitly expands to \code|a => (b => c)|.
}
We adopt a convention of collecting parameters on the left side of the ``\code| ⇒ |'' and combining their type annotation if they are shared.
So we could more simply write the following.
%
\begin{snippet}
(b1 b2 : boolean) ⇒
  if b1
    then if b2 then true else false
    else false
\end{snippet}
%

However, by itself, a function is inert.
In order to \ep{do} something, we need to provide it with arguments.
A \kw{term application} is a term that encodes the application of a function to an argument.
To write the application of a function $f$ to an argument $a$, we write \code|$f$ $a$|.
For example, to apply the function \code|not| to the argument \code|true|, we write \code|not true|.
As for applying functions of multiple parameters, application is left-associative%
\footnote{
  I.e. \code|a b c| implicitly expands to \code|(a b) c|
}
so arguments can simply be written in sequence after the function.
For example, to apply the function \code|and| to two arguments is written
%
\begin{snippet}
and true false
\end{snippet}
%
which implicitly associates to
%
\begin{snippet}
(and true) false
\end{snippet}
%
In other words, \code|and true| evaluates to a function of one parameter, and \code|false| is provided as that last parameter.
This style of writing functions of multiple arguments is called \ep{currying}.

There are two sorts of types that have specially constructed terms that we will be using often: \ep{sum types} and \ep{product types}.
Firstly, a \kw{sum type} is a type with terms that must be constructed via a few defined cases.
For example, the type \code|boolean| is a sum type who's terms have two atomic cases \code|true| and \code|false|.
We can declare such a sum type with the following syntax.
%
\begin{snippet}
type boolean : kind
  { true  : boolean
  | false : boolean }.
\end{snippet}
%
The ``\code` | `'' divides the two cases.
Note that the ``\code|boolean : kind|'' phrase declares that \code|boolean| is a type with kind \code|kind|.
The details of types and kinds will be explained in the next section.

We can also use the control structure \code|cases| in order to break a term of a sum type into to cases.
For example, we could declare the function \code|and| using \code|cases| as follows.
%
\begin{snippet}
term and (b1 b2 : boolean) : boolean
  ≔ cases b1
      { true  ⇒ cases b2
                  { true  ⇒ true
                  | false ⇒ false }
      | false ⇒ false }.
\end{snippet}
%
Another useful pattern, especially for sum types, is that of \ep{recursive} definitions i.e. a term who's definition refers to itself.
Though this can sometimes result in functions that do not terminate during evaluation, we shall only consider recursive functions that do terminate i.e. \ep{total} functions.
A classic \ep{inductive} type is that of the natural numbers, called inductive because it is recursively structured and has a base case.
It is a sum type that can be defined as follows.
%
\begin{snippet}
type natural : kind
  { zero : natural
  | succ : natural → natural }.
\end{snippet}
%
The \code|zero| case of \code|natural| encodes the natural number $0$.
The \code|succ| case of \code|natural| encodes a function that evaluates to the successor of its parameter $n$.
We can define a recursive function \code|add| for natural numbers as follows, where the sum of \code|zero| and \code|n| is \code|n|, and the sum of \code|succ m'| and \code|n| is the successor of the sum of \code|m'| and \code|n|.
%
\begin{snippet}
term add (m n : natural) : natural
  ≔ cases m
      { zero    ⇒ n
      | succ m' ⇒ succ (add m' n) }.
\end{snippet}
%
Soon, in section~\ref{sec:LangA-reduction}, reduction will be introduced which will allow us to actually carry out the computations encoded by terms such as \code|add|.

There is a generic type-polymorphic sum type constructor, \code|$α$ ⊕ $β$|, which expresses the sum type of types $α$ and $β$.
The term constructor for the $α$-case is \code|left| and for the $β$-case is \code|right|.

Secondly, a \kw{product type} is a type with terms that must be constructed using terms of each of a collection of \ep{component} types.
For example, the type of pairs of integers, \code|integral-pair| is the product type of two copies of \code|integer|, written as follows:
%
\begin{snippet}
type integral-pair : kind
  { x : integer
  ; y : integer }.
\end{snippet}
%
The two integer components, \code|x| and \code|y| are named and separated by a ``\code| ; |''.
These components names can be used to access the corresponding value of a term of the product type, where \code|x| and \code|y| are functions \code|integral-pair → integer|.
For example, if \code|p| has type \code|integral-pair|, then \code|x p| evaluates to the $x$-component and \code|y p| evaluates to the $y$-component.

There is also a generic type-polymorphic product type constructor, \code|$α$ ✕ $β$|.
The type constructor for \code|$α$ ✕ $β$| is \code|($a$, $b$)| where \code|a| is a term of type $α$ and $b$ is a term of type $β$.
Using this constructor for terms of a type product is common in the arguments of functions as well.
For example, the following function sums the two components of a \code|integral-pair|-term.
%
\begin{snippet}
term integral-pair-sum ((x, y) : integral-pair) : integer ≔ x + y.
\end{snippet}
%

Lastly, there is a parallel correspondence between the syntax of terms and of types.
There is a \ep{type atom}, \ep{type function}, and \ep{type application}, which have the same structure of the term versions.
However, the term and type syntax levels cannot intermingled, for example a term application cannot evaluate to a type.
\kw{Type functions} are used to make a \kw{type-polymorphic} --- that is, be generalized over a kind of types.
For example, the \code|list| type is type-polymorphic because it can be applied to a specific types to be the type of lists of integers, the type of lists of booleans, etc.
The type-parameter \code|α:kind| is a sort of ``for all \code|α| of kind \code|kind|''.
The type \code|list| is formed as a recursive type function as follows.
%
\begin{snippet}
type list : kind ≔ (α:kind) ⇒ unit ⊕ (α ✕ list α)
\end{snippet}
%
where \code|unit| is the type that has exactly one term of it, \code|•|.
A list of $α$ is either an empty list, represented by \code|•|, or list head $α$-term and a list tail \code|(list $α$)|-term.
We could also write \code|list| in the more convenient named sum type notation, where the empty list is named \code|nil| and constructor appending an $α$-term to a tail list is named \code|app|.
%
\begin{snippet}
type list (α:kind) : kind
  { nil : list α
  | app : α → list α → list α }.
\end{snippet}
%
As \code|list| is such a common data structure, we shall adopt a few notations for it: \code|[]| abbreviates \code|nil| and \code|$a$∷$as$| abbreviates \code|app $a$ $as$|.
Additionally a longer list can be written using the notation \code|[$a_1$, $\dots$, $a_n$]| which abbreviates \code|$a_1$∷ $\cdots$ ∷$a_n$∷[]|.

Notice that since \code|list| is parametrized by a type $α$, the type of \code|app| is \code|α ⇒ α → list α → list α|.
It can be easy to get confused by the mixture of arrow-like infixed operators.
When applying \code|app|, the argument provided in place for \code|α| can be provided explicitly or inferred by the other arguments.
For example, \code|app integer 1 (nil integer)| provides \code|integer| explicitly as a first argument to \code|app| and \code|nil| to fill in for \code|α|, requiring the other arguments to match \code|α $=$ integer|.
However, we could equivalently write \code|app 1 nil|, since the argument \code|integer| filling in for \code|α| is inferred from the fact that \code|1| is an integer.
The latter version will commonly be adopted for the sake of conciseness.

There are a few other data structures that will be introduced along the way in this thesis, and we shall adopt a few other notations.
But, ultimately, they all are expressible in terms of the basic syntax presented in this section.
See \prelude{\LangA} for more detailed examples of basic \LangA programs.

% ------------------------------------------------------------------------------
\subsection{Typing}
\label{sec:LangA-typing}

Terms and types are related by a \kw{typing judgement}, by which a term is stated to \ep{be of} or \ep{have} a type.
The judgement that term $a$ has type $α$ is written as \code|$a$:$α$|.
Similarly, types and kinds are related by a \kw{kinding judgement}, where the judgement that type $α$ has kind $K$ is written \code|$α$:$K$|.
(Kinding will be much less central to our considerations however, so we refer to typing judgements as also implicitly including kinding judgements.)

A well-formed (following syntax rules) term may or may not be well-typed (following typing rules).
Observe that \code|not true| is well-formed and well-typed, since \code|not| is a function with a \code|boolean|-parameter and \code|true| is a \code|boolean|-term supplied as the application's argument.
Observe also that \code|not 1| is well-formed but \ep{not} well-typed, since \code|not| has a \code|boolean|-parameter yet \code|1| is an \code|integer|-term.
The checking that terms have properly-corresponding types, as will be detailed in this section, is called \kw{type-checking} or just \kw{typing}.

In order to build typed terms using the generators presented in~\ref{tab:LangA-syntax}, the types of complex terms are inferred from their sub-terms via inference rules within a \ep{judgement context}, where a \kw{judgement context} is a collection of typing (and kinding) judgements.
Here, ``inference'' correlates with the intuitive concept, but the \ep{only} inferences allowed are those exactly corresponding to the collection of inference rules provided --- these are pure formalisms.
A proposition of the form \code|$\jctx$ $⊢$ $a$:$α$| asserts that the context $\jctx$ entails \code|$a$:$α$|.
If a particular judgement $J$ from the judgement context is required in an entailment's premise, then the notation \code|$\jctx$, $J$ $⊢$ $J'$| abbreviates \code|$\jctx \cup \set{J}$ $⊢$ $J'$|.

With judgements, we now can state the \kw{typing inferences rules} for \LangA.
An inference rule has form
%
\begin{display}
\begin{tabular}{l}
  $P_1$ \iand $\cdots$ \iand $P_m$
\\\hline
  $Q_1$ \iand $\cdots$ \iand $Q_n$
\end{tabular}
\end{display}
%
which asserts that the premises $P_1, \dots, P_m$ entail the conclusions $Q_1, \dots, Q_n$.
For example,
%
\begin{display}
\begin{tabular}{l}
  $\jctx$ is a judgement context \\
  $a$ is a term \\
  $α$ is a term \\
  $K$ is a kind \\
  \code|$\jctx$ $⊢$ $a$:$α$| \\
  \code|$\jctx$ $⊢$ $α$:$A$|
\\\hline
  \code|$\jctx$ $⊢$ $a$:$α$| \\
  \code|$\jctx$ $⊢$ $α$:$A$|
\end{tabular}
\end{display}
%
could be a particularly uninteresting inference rule.
Explicitly stating the domain of each name in the premises is cumbersome however, so the the domains of names are adopted based on their names:
%
\begin{itemize}
  \item $\jctx$ is a judgement context,
  \item $a, b, c, \dots$ are terms,
  \item $α, β, γ, \dots$ are types,
  \item $A, B, C, \dots$ are kinds.
\end{itemize}
%
Additionally, unless otherwise stated, a type is presumed to have kind \code|kind|.

\newpage
The typing rules for \LangA are the following.
%
\begin{table}[H] \centering
\captionof{table}{Typing in \LangD: Resource}
\label{tab:LangD-typing-resource}
\begin{tabular}{rp{25em}}
  \rulename{Resource} &
  \begin{tabular}{p{25em}}
    \iforall \code`$Γ$ $⊢$ $α_i$ : Type` \\
    \iforall \code`$Γ$ $⊢$ $β_i$ : Type` \\
    \code`resource $ρ$ { $g_1$ : $α_1$ ↗ $β_1$ ; $\cdots$ ; $g_n$ : $α_n$ ↗ $β_n$ }.`
    %
  \\ \hline
    %
    \code`$Γ$ $⊢$ $ρ$ : Resource` \\
    \iforall \code`$Γ$ $⊢$ $g_1$ : $α_1$ ↗ $β_1$`
  \end{tabular}
\end{tabular}
\end{table}

\begin{table}[H] \centering
\captionof{table}{Typing in \LangD: Channel}
\label{tab:LangD-typing-channel}
\begin{tabular}{rl}
  \rulename{Channel} &
  \begin{tabular}{p{10em}}
    \code`$Γ$ $⊢$ $ρ$ : Resource` \\
    \code`channel $c$ : $ρ$.`
  \\ \hline
    \code`$c$ : $ρ$`
  \end{tabular}
\end{tabular}
\end{table}

\begin{table}[H] \centering
\captionof{table}{Typing in \LangD: Performance}
\label{tab:LangD-typing-performance}
\begin{tabular}{rp{25em}}
  \rulename{Perform} &
  \begin{tabular}{p{25em}}
    \code`$Γ$ $⊢$ $ρ$ : Resource` \iand
    \code`$Γ$ $⊢$ $c$ : $ρ$` \\
    $\provprop{ρ}{g}$ \iand
    \code`$Γ$ $⊢$ $g$ : $α$ ↗ $β$` \iand
    \code`$Γ$ $⊢$ $a$ : $α$` \\
  \hline
    \code`$Γ$ $⊢$ $c$#$g$ $a$ : $β$`
  \end{tabular}
\end{tabular}
\end{table}

\begin{table}[H] \centering
\captionof{table}{Typing in \LangD: Handler}
\label{tab:LangD-typing-handler}
\begin{tabular}{rl}
  \rulename{Handler} &
  \begin{tabular}{p{25em}}
    \code`$Γ$ $⊢$ $ρ$ : Resource`
    \\
    \iforall $\provprop{ρ}{g_i}$
    \\
    \iforall \code`$Γ$ $⊢$ $g_i$ : $χ_i$ ↗ $υ_i$`
    \\
    \iforall \code`$Γ$, $x_i$:$χ_i$, $k_i$:$υ_i$ → $β$ $⊢$ $b_i$:$β$`
    \\
    \code`$Γ$ $⊢$ $a_v$:$α$` \iand
    \code`$Γ$ $⊢$ $b_v$:$β$`
    \\
    \code`$Γ$ $⊢$ $b_f$:$β$` \iand
    \code`$Γ$ $⊢$ $c_f$:$γ$`
    \\
  \hline
    \code`$Γ$ $⊢$ handler` \\
    \code`      { #$g_1$ $x_1$ $k_1$ ⇒ $b_1$ ; $\cdots$ ; #$g_n$ $x_n$ $b_n$ ⇒ $y_n$` \\
    \code`      ; value $a_v$ ⇒ $b_v$` \\
    \code`      ; final $b_f$ ⇒ $c_f$ }` \\
    \code`      : $ρ$ → $α$ ↘ $γ$`
  \end{tabular}
\end{tabular}
\end{table}

\begin{table}[H] \centering
\captionof{table}{Typing in \LangD: Handling}
\label{tab:LangD-typing-handling}
\begin{tabular}{rp{25em}}
  \rulename{Handling} &
  \begin{tabular}{p{20em}}
    \code`$Γ$ $⊢$ $p$:$α$` \iand
    \code`$Γ$ $⊢$ $h$ : $α$ ↘ $β$` \\
  \hline
    \code`$Γ$ $⊢$ $p$ with $h$ : $β$`
  \end{tabular}
\end{tabular}
\end{table}



Though typing rules will be presented for each new syntactical structure in later chapters, in this work they serve as a background assurance rather than a main feature of study.
All expressions from here on will be assumed well-typed.
% For the sake of explanation however, consider the following program.
For the sake of explanation however, the following demonstrates the typing of the term \code|[not true]|, which forms a proof that the term is indeed well-typed.
Let $Γ$ be the judgement context yielded by the declarations provided so far in this chapter.
%
\begin{display}
\begin{tabular}{l}
  \begin{tabular}{l}
    \code|$Γ$ $⊢$ not : boolean → boolean| \\
    \code|$Γ$ $⊢$ true : boolean|
  \\\hline
    \code|$Γ$ $⊢$ not true : boolean|
  \end{tabular}
  \iand
  \begin{tabular}{l}
    \code|$Γ$ $⊢$ nil : (α:kind) ⇒ list α| \\
    \code|$Γ$ $⊢$ boolean : kind|
  \\\hline
    \code|$Γ$ $⊢$ nil boolean : list boolean|
  \end{tabular}
\\[2em]
  \code|$Γ$ $⊢$ app : (α:kind) ⇒ α → list α → list α|
\\[0.75em]
  \code|$Γ$ $⊢$ boolean : kind|
\\[1em]\hline\\
  \code|$Γ$ $⊢$ app boolean (not true) (nil boolean) : list boolean|
\end{tabular}
\end{display}
%

% ------------------------------------------------------------------------------
\newpage
\subsection{Reduction}
\label{sec:LangA-reduction}

Finally, the last step is to introduce \kw{reduction}.
So far we have outlined syntax and typing for building well-formed, well-typed expressions in \LangA, but all these expressions are static.
Reduction rules describe how terms can be transformed, step by step, in a way that models computation.
The reduction of a term $a$ to a term $a'$ is written $a \rto a'$.
A series of these simple reductions may end in a term for which no reduction rule can apply. Call these terms \kw{values}, and notate ``$v$ is a value'' as ``$\valprop{v}$.''
To reduce a term via a series reduction-rule-applications that ends in a value is called \kw{evaluation}.

The following reduction rules are given for \LangA:
%
\begin{table}[H] \centering
\captionof{table}{Reduction in \LangA}
\begin{tabular}{rp{25em}}
  \rulename{Simplify} &
  \begin{tabular}{p{10em}}
    $a$ $\rto$ $a'$ \\
  \hline
    \code|$a$ $b$| $\rto$ \code|$a'$ $b$|
  \end{tabular}
\\[2em]
  \rulename{Simplify} &
  \begin{tabular}{p{10em}}
    $b$ $\rto$ $b'$ \iand
    $\valprop{v}$ \\
  \hline
    \code|$v$ $b$| $\rto$ \code|$v$ $b'$|
  \end{tabular}
\\[2em]
  \rulename{Apply} &
  \begin{tabular}{p{20em}}
    $\valprop{v}$
    \\ \hline
    \code|(($a$ : $α$) ⇒ $b$) $v$| $\rto$ \code|$[v/a]$ $b$|
  \end{tabular}
\\[2em]
  \rulename{Split-Left} &
    \code|split (left $a$) $f$ $g$|
    $\rto$
    \code|$f$ $a$|
\\[2em]
  \rulename{Split-Right} &
    \code|split (right $b$) $f$ $g$|
    $\rto$
    \code|$f$ $b$|
\\[2em]
  \rulename{Project-First} &
    \code|first ($a$, $b$)|
    $\rto$
    \code|$a$|
\\[2em]
  \rulename{Project-Second} &
    \code|second ($a$, $b$)|
    $\rto$
    \code|$b$|
\end{tabular}
\label{tab:LangA-reduction}
\end{table}


The most fundamental of these rules is \rulename{Apply} (also known as $β$-reduction),
which is the way that function applications are resolved to
the represented computation's output.
The substitution notation $[v/a]b$ indicates to
``replace with $v$ each appearance of $a$ in $b$.''
In this way, for a function \code|(($a$:$α$) ⇒ $b$) : $α$ → $β$| and
an input \code|$v$:$α$|,
$β$-Reduction \kw{substitutes} the input $v$ for the appearances of the function parameter $a$ in the function body $b$.

For example, the type \code|boolean| is defined as a type sum.
%
\begin{snippet}
type boolean : kind ≔ unit ⊕ unit.
\end{snippet}
The \code|left •| case is abbreviated by \code|true| and the \code|right •| case is abbreviated by \code|false|.
Then consider the following definition of \code|not|.
%
\begin{snippet}
term not (b:boolean) : boolean
  ≔ cases b
      { left  _ ⇒ false   // don't need the unit term parameter,
      | right _ ⇒ true }. // so _ indicates an unused parameter
\end{snippet}
%
The \code|cases| structure is actually a notation that the above into the following.
%
\begin{snippet}
term not : boolean → boolean
  ≔ (b:boolean) ⇒ split b (_ ⇒ false) (_ ⇒ true)
\end{snippet}
%
Then we can reduce an application \code|not (left •)| as follows.
%
\begin{snippet}
not (left •)
$↠$
split (left •) (_ ⇒ false) (_ ⇒ true)
$↠$ $\ruleapp{Split-Left}$
(_ ⇒ false) •
$↠$ $\ruleapp{Apply}$
false
\end{snippet}

As another example, recall the function \code|add : natural → natural → natural|.
The addition of \code|N2 $:=$ succ (succ zero)| to \code|N3 $:=$ succ (succ (succ zero))| is computed as follows.
%
\begin{snippet}
add N2 N3
$↠$ $\ruleapp{Definition}$
(m n ⇒ cases m{ zero ⇒ n | succ m' ⇒ succ (add m' n) })
  (succ(succ zero)) N3
$↠$ $\ruleapp{Apply}$
cases (succ(succ zero)){ zero ⇒ N3 | succ m' ⇒ succ (add m' N3) }
$↠$ $\ruleapp{Split-Right}$
(m' ⇒ succ(add m' N3)) (succ zero)
$↠$ $\ruleapp{Definition}$
(m' ⇒ succ((m n ⇒ cases m{zero ⇒ n | succ m' ⇒ succ(add m' n)}) m' N3))
  (succ zero)
$↠$ $\ruleapp{Apply}$
succ((m n ⇒ cases m{ zero ⇒ n | succ m' ⇒ succ(add m' n) }) (succ zero) N3)
$↠$ $\ruleapp{Apply}$
succ(cases (succ zero){ zero ⇒ N3 | succ m' ⇒ succ(add m' N3) })
$↠$ $\ruleapp{Split-Right}$
succ(m' ⇒ succ(add m' N3) zero)
$↠$ $\ruleapp{Definition}$
succ(succ((m n ⇒ cases m{ zero⇒n | succ m' ⇒ succ (add m' n) }) zero N3))
$↠$ $\ruleapp{Apply}$
succ(succ((cases zero{ zero ⇒ N3 | succ m' ⇒ succ (add m' N3) })))
$↠$ $\ruleapp{Split-Left}$
succ(succ((_ ⇒ N3) •)) // the • comes from zero = left •
$↠$
succ(succ N3)
$=:$
N5
\end{snippet}

% ------------------------------------------------------------------------------
\subsection{Properties}
% usefule source:
% https://web-static-aws.seas.harvard.edu/courses/cs152/2016sp/lectures/lec11-types.pdf
With the syntax, typing, and reduction defined for \LangA,
we now have a completed definition of the language.
However, some of the design decisions may seem arbitrary.
This particular framework is good because it maintains a few nice properties that make reasoning about \LangA intuitive and extendable.
Here when referring to a \ep{term} there is the implicit premise that the term is well-formed.

\begin{theorem}
  \kw{(Type-Preserving Substitution in \LangA)}.
  If
    \code|$\jctx$, $a$:$α$ $⊢$ $b$:$β$|,
    \code|$\jctx$ $⊢$ $v$:$α$|, and
    $\valprop{v}$,
  then
    \code|$\jctx$ $⊢$ $([v/a]b)$:$β$|.
  I.e. the \rulename{Apply} reduction rule always results in a term with the type of the function's output.
\end{theorem}

\begin{theorem}
  \kw{(Reduction Progress in \LangA)}.
  If
    \code|$\set{}$ $⊢$ $a$:$α$|,
  then
    either $\valprop{a}$ or
    there exists a term $a'$ such that $a \rto a'$.
  I.e. well-typed terms are either values or reducible.
\end{theorem}

\begin{theorem}
  \kw{(Type Preservation in \LangA)}.
  If
    \code|$\jctx$ $⊢$ $a$:$α$| and $a \rto a'$,
  then
    \code|$\jctx$ $⊢$ $a'$:$α$|.
  I.e. reducing a well-typed term always results in a well-typed term of the same type.
\end{theorem}

% NOTE: is redundant to type preservation
% \begin{theorem}
%   \kw{(Type Soundness in \LangA)}.
%   If
%     \code|$\jctx$ $⊢$ $a$:$α$ and $a \rto a'$|,
%   then
%
%
%     either
%       $\valprop{a'}$ and \code|$\jctx$ $⊢$ $a'$,
%     or
%       there exists a term $a''$ such that
%         \code|$\jctx$ $\entails$ $a''$:$α$| and $a' \rto a''$.
%   I.e. at every step of reduction
%
%   , a term has the same type.
% \end{theorem}

\begin{theorem}
  \kw{(Strong Normalization in \LangA)}.
  For any term $a$,
  either
    $\valprop{a}$
  or
    there is a sequence of reductions from $a$ that ends in a term $a'$ such that $\valprop{a'}$.
  I.e. every term evaluates in a finite number of steps.
\end{theorem}
