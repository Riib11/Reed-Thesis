\chapter*{Abstract}
There exists a dilemma of two programming language design philosophies: \ep{imperative} languages and \ep{declarative} languages.
Imperative languages are well-adapted for real-world programming, one reason being their ability to interact with \ep{implicit contexts}, which is formalized by the concept of \ep{performing effects}.
Though effects can be very useful, they are often \ep{dangerous} --- effects yield complexly-interdependent behaviors that are hard to analyze and predict.
Declarative languages are less well-adapted for this imperative style of programming, but can be much more \ep{safe} --- behavior is locally contained and easily analyzable and predictable.

This thesis approaches the challenge of designing a declarative programming language that provides effects with all the same capabilities of imperative language's effects while still maintaining safety.
The design process follows a progression of five languages:
(1) \LangA is a foundational declarative language,
(2) \LangB extends \LangA with imperative effects,
(3) \LangC extends \LangA with monadic effects,
(4) \LangD extends \LangA with algebraic effects with handlers, and finally
(5) \LangE extends \LangC with a freer-monadic effects (a monadic implementation of algebraic effects with handlers).
The effect framework of \LangE meets the goals of the design challenge, and a subsequent discussion analyzes its significance.
