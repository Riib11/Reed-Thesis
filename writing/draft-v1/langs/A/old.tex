% ------------------------------------------------------------------------------
% ------------------------------------------------------------------------------
\subsection{Syntax for \LangA}

In \LangA, there are two forms used for constructing well-formed expressions: \kw{terms} and \kw{types}.
They are expressed by the following syntax:

\begin{table}[H]
\label{tab:LangD-syntax}
\centering
\captionof{table}{Syntax for \LangD}
\begin{tabular}{r|L{25em}|l}
  \kw{metavariable} & \kw{generator} & \kw{name}
\\ \hline \hline && \\
  \mvar{declaration}
    & \code`resource $\mvar{resource-name}$ $〚$ ($\mvar{type-param}$:$\mvar{kind}$) $〛$`
    & resource \\
    % & \code`           $〚$ ($\mvar{type-param}$:$\mvar{kind}$) $〛$` \\
    & \code`  { $〚$ $\mvar{action-name}$ $〚$ ($\mvar{type-param}$:$\mvar{kind}$) $〛$`
    & definition \\
    & \code`        : $\mvar{type}$ ↗ $\mvar{type}$ ; $〛$ }.`
  \\[1em]
    & \code`channel $\mvar{channel-name}$ : $\mvar{type}$.`
    & channel \\&& instantiation
\\[1em] \hline && \\
  \mvar{kind}
    & \code`Resource`
    & resource kind
\\[1em] \hline && \\
  \mvar{type}
    & \code`$\mvar{type}$ ↗ $\mvar{type}$`
    & action type
  \\[1em]
    & \code`$\mvar{type}$ ↘ $\mvar{type}$`
    & handler type
\\[1em] \hline && \\
  \mvar{term}
    & \code`$\mvar{channel-name}$#$\mvar{action-name}$ $\mvar{term}$`
    & performance
  \\[1em]
    & \code`handler $\mvar{term-name}$ $〚$ ($\mvar{type-param}$:$\mvar{kind}$) $〛$`
    & handler \\
    & \code`  : $\mvar{kind}$` \\
    & \code`  { $〚$ #$\mvar{action-name}$ $\mvar{term-parap}$ $\mvar{term-param}$` \\
    & \code`        ⇒ $\mvar{term}$ ; $〛$ }`
  \\[1em]
    & \code`$\mvar{term}$ with $\mvar{term}$`
    & handle \\&& performance
\\[1em]
\end{tabular}
\end{table}

% action
% perform
% do with

% resource $\mvar{resource-name}$ $〚$ ($\mvar{type-param}$:$\mvar{kind}$) $〛_*$ : Resource
%   { $\mvar{action-name}_1$ $〚$ ($\mvar{type-param}$:$\mvar{kind}$) $〛_1$ : $\mvar{type}_{1,\text{in}}$ ↗ $\mvar{type}_{1,\text{out}}$
%   ; $\cdots$
%   ; $\mvar{action-name}_n$ $〚$ ($\mvar{type-param}$:$\mvar{kind}$) $〛_n$ : $\mvar{type}_{n,\text{in}}$ ↗ $\mvar{type}_{n,\text{out}}$ }.


% ------------------------------------------------------------------------------
\subsubsection{Metavariables and Names}
The metavariables $\mvar{term-name}$, $\mvar{type-name}$, and $\mvar{term-param}$ range over, for a given program, a fixed and collection of names.
There are two kinds of names included in this collection:
\begin{itemize}
\item \kw{constructed names}: Require a construction written in the language; establish a \ep{functional equality} between name and construction --- the name universally can be substituted \ep{with} its construction.
\item \kw{basic names}: Require a construction written in the language; establish a \ep{definitional equality} between name and construction --- the name universally can be substituted \ep{with} or \ep{for} its construction.
\item \kw{primitive names}: Included a priori. Terms and types provided this way are provided along with their a priori type or kind respectively, and primitive terms along with a priori reduction rules (if any). Primitive names are also trivially basic.
\end{itemize}

This syntax schema is formatted as a \ep{generative context-free grammar}, with
\begin{itemize}
  \item
  \ep{non-terminals}: The meta-variables \mvar{progam}, \mvar{declaration}, \mvar{kind}, \mvar{kind}, \mvar{term}.
  \item
  \ep{terminals}: Ranged over by the meta-variables \mvar{term-name}, \mvar{type-name}.
  \item
  \ep{repeated}: Expressions of the form $[~\mvar{meta-var}~]$ indicate that any number of \mvar{meta-var} can be repeated in its place.
  Note also that metavariables inside of repeated expressions may be indexed by $i$ or $j$, indicating that they are the $i$-th repeated sub-expression.
\end{itemize}

The following are some examples of the sorts of formal items that these meta-variables stand range over:
\begin{itemize}
  \item \mvar{term-name}:
    \code|1|,
    \code|12309|,
    \code|true|,
    \code|false|,
    \code|"hello world"|,
    \code|$\unitterm$|.
  \item \mvar{type-name}:
    \code|natural|,
    \code|integer|,
    \code|boolean|,
    \code|string|,
    \code|void|,
    \code|unit|.
  \item \mvar{term-param}:
    \code|x|,
    \code|y|,
    \code|this-is-a-parameter|.
\end{itemize}
\TODO{include some function terms and (and maybe types?)}

A given expression is well-formed with respect to \LangA if it is constructible by a series of applications of these rules.
Note that this syntax does not specify any concrete terms or types.
In this way, \LangA's syntax is defined so abstractly in order to make the definitions for its typing and semantics as concise as possible.
For example, we shall reference to the type \code|int|, some \mvar{term-name}s that have type \code|int| such as
\code|0|, \code|1|, \code|-1|, \code|2|, \code|-2|, \code`3`, \code`-3`,
and so on.
Such types and terms as these are called \kw{atoms} because they do not have any internal; they are simply posited as primitives in the language.
