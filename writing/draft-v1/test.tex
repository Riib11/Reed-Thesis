\documentclass[12pt,twoside]{reedthesis}
\usepackage{main}

% \usepackage[artemisia]{textgreek}

% \usepackage{etoolbox}
% \usepackage{xpatch}

% \makeatletter
% \patchcmd{\lsthk@TextStyle}{\let\lst@DefEsc\@empty}{}{}{\errmessage{failed to patch}}
% \makeatother

% \renewcommand{\code}[1]{
%   \lstinline[mathescape=false, escapechar={$}, escapebegin={\(}, escapeend={\)}]|#1|
% }

% \renewcommand{\code}{
%   \lstinline[mathescape=false, escapechar={$}, escapebegin={\(}, escapeend={\)}]}

% \renewcommand{\codem}{
%   \mbox\lstinline[mathescape=false, escapechar={$}, escapebegin={\(}, escapeend={\)}]}


% -----------------------------------------------------------------------------
% HEAD
% -----------------------------------------------------------------------------
% \title{Pure Effects}
% \author{Henry Blanchette}
% \date{}
% \division{Mathematics and Natural Sciences}
% \department{Computer Science}
% \advisor{Jim Fix}
% -----------------------------------------------------------------------------
% TITLE
% ----------------------- ------------------------------------------------------
\begin{document}

\chapter{Example}

% math: $(α)$.
% code: \code|($\text{α}$)|.

Hello world.
$α$

\textsf{\SourceCodePro this is SourceCodePro α}

Hello {\SourceCodePro (α)} world {\SourceCodePro (β)}.

\begin{lstlisting}[extendedchars=false, basicstyle={\SourceCodePro}, fontadjust=false]
  (α)(β)(ψ)(δ)
\end{lstlisting}

\begin{program}
  ηελλο ςορλ(δ)
\end{program}

% hello \char"2012 world

% % ------------------------------------------------------------------------------
% % ------------------------------------------------------------------------------
% % ------------------------------------------------------------------------------
%
% \section{Language \LangA}
% \label{sec:LangA}
%
% \lc is a formal language for expressing computation.
% In this context, a \kw{language} is defined to be a set of expression that are generated by syntactical rules.
% The \lc comes in many different variants, and here we will consider a basic variant of the \kw{simply-typed \lc} in order to considering computation formally. Call our language \LangA.
%
% % ------------------------------------------------------------------------------
% \subsection{Syntax for \LangA}
%
% In \LangA, there are two forms used for constructing well-formed expressions: \kw{terms} and \kw{types}.
% They are expressed by the following syntax:
% \begin{align}
% \label{syn:LangA}
% \begin{syntaxarray}
% \mvar{program}
%   & [~ \mvar{declaration}~ ] &
%     \text{program}
% \\ \hline
% \mvar{declaration}
%   % & \code{primitive term $\mvar{term-name}$ : $\mvar{type}$.} &
%   % & \mbox{\code{primitive \$term\$ term-name : type.}} &
%   % & \codem{term $\hspace{0.5em}\mvar{term-name}$ := $\hspace{0.5em}1$} &
%   %   \text{primitive term}
%   % \\
%   % & \code{term $\mvar{term-name}$ : $\mvar{type}$ ≔ $\mvar{term}$.} &
%   %   \text{constructed term}
%   % \\
%   % & \code{primitive type $\mvar{type-name}$ : $\mvar{Type}$.} &
%   %   \text{primitive type}
%   % \\
%   % & \code{type $\mvar{type-name}$ : $\mvar{Type}$ ≔ $\mvar{type}$.} &
%   %   \text{constructed type}
% \\ \hline
% % \mvar{Type}
% %   & \code{Type} &
% %     \text{atom}
% %   \\
% %   & \code{Type → $\mvar{Type}$} &
% %     \text{arrow}
% % \\ \hline
% % \mvar{type}
% %   & \mvar{type-name} &
% %     \text{atom}
% %   \\
% %   & \code{($\mvar{type-param}$ : $\mvar{Type}$) ⇒ $\mvar{type}$} &
% %     \text{function}
% %   \\
% %   & \code{$\mvar{type}$ $\mvar{type}$} &
% %     \text{application}
% % \\ \hline
% % \mvar{term}
% %   & \mvar{term-name} &
% %     \text{atom}
% %   \\
% %   & \code{($\mvar{term-param}$ : $\mvar{type}$) ⇒ $\mvar{term}$} &
% %     \text{function}
% %   \\
% %   & \code{$\mvar{term}$ $\mvar{term}$} &
% %     \text{application}
% \end{syntaxarray} \end{align}
%
% % \begin{align} \begin{array}{c}
% %   \begin{array}{l}
% %   \code{\codekey{case} $\mvar{term}$ to $\mvar{type}$} \\
% %   \code{  \{ left ($\mvar{term-param}_1$:$\mvar{type}_1$) ⇒ $\mvar{term}_1$} \\
% %   \code{  | right ($\mvar{term-param}_2$:$\mvar{type}_2$) ⇒ $\mvar{term}_2$ \}}
% %   \end{array} \vspace{0.3em}
% %   \\ \syneq \\
% %   \begin{array}{l}
% %   \code{split $\mvar{type}_1$ $\mvar{type}_2$ $\mvar{type}$} \\
% %   \code{  (($\mvar{term-param}_1$:$\mvar{type}_1$) ⇒ $\mvar{term}_1$)} \\
% %   \code{  (($\mvar{term-param}_2$:$\mvar{type}_2$) ⇒ $\mvar{term}_2$)}
% %   \end{array}
% % \end{array} \end{align}
%
% % \code{term hello world ≔ $x$}
%
% % \[
% % %   % \mbox{\lstinline[mathescape=true]|primitive $x$ := $y$.|}
% %   \text{\code|primitive x ≔ y.|}
% % \]
%
%
%
% \begin{lstlisting}[mathescape=false, escapechar=$, escapebegin={begin}, escapeend={end}]
%   this is an $interesting$ test
% \end{lstlisting}
%
% % \[
% %   \mbox{\code|primitive $t$ $\mvar{term-name}$ : type.|}
% % \]
%
% Consider the following equation:
% % \codem{term hello world ≔ $x$}.
% This is important because..
%
%
% \begin{lstlisting}[mathescape=false, escapechar=$, escapebegin={\[}, escapeend={\]}]
%   hello world $α$
% \end{lstlisting}
%
% \lstinline[mathescape=false, escapechar=$, escapebegin={\(}, escapeend={\)}]!hello $≔$ to $α$!
%
% \lstinline[mathescape=false, escapechar=$, escapebegin={\(}, escapeend={\)}]!hello ≔ to $α$!
%
% \code!hello ≔ $abc$!
% \code`hello ≔ $abc$`
%
% % \code{term hello ≔ to α}
%
% % \code{hello ≔ $α$}
% % \code{hello = $α$}

\end{document}
