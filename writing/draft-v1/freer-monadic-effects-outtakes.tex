% -----------------------------------------------------------------------------
% -----------------------------------------------------------------------------
% -----------------------------------------------------------------------------
\section{Functor}
\label{sec:definition-of-functor}

In section~\ref{sec:demo-monad}, we described the mapping capability of a the mutability monad.
Though they are not necessarily monads, it turns out that the class of structures with at least the mapping capability are interesting on their own as well.
We shall formalize this class here as part of the process of formalizing monad, and also for useful future reference.
\begin{blockdefinition}
A \kw{functor} is a structure with a mapping capability i.e. functors can be mapped over.
\end{blockdefinition}
Since monads must also have such a mapping capability, all monads are also functors.

% -----------------------------------------------------------------------------
% -----------------------------------------------------------------------------
\subsection{The Functor Type-class}
\label{sec:functor-type-class}

Functors are data structures that can be mapped over.
In more type-oriented terminology, a functor is a type \code`$Φ$ : Type → Type` that has a capability \code`$map$ : α β ⇒ (α → β) → (Φ α → Φ β)`.
The following definition of \code`Functor` captures this specification in the type-class notation.
\begin{program}[caption={Definition of the type-class \code|Functor|}]
class Functor (Φ : Type → Type) : Type
  { map (α β : Type) : (α → β) → (Φ α → Φ β) }.
\end{program}

% -----------------------------------------------------------------------------
% -----------------------------------------------------------------------------
\subsection{Free Monad}

\TODO{do I actuallly want to include this section? Maybe I can just describe it without going into detail... can mention issues.}

\TODO{use standardized type-class instance parameter notation, defined in monadic-effects chapter}

\begin{program}[caption={free monad}]
type Free (φ : Type → Type) (α : Type) : Type
  ≔ pure   : α → Free φ α
  | impure : φ (Free φ α) → Free (φ α).

% type Free (φ : Type → Type) (α : Type) : Type
%   ≔ α + φ (Free φ α)
%
% term pure (φ α : Type) : α → Free φ α ≔ left.
% term impure (φ α : Type) : φ (Free φ α) → Free (φ α) ≔ right.

term lift-Free (φ α : Type)
  : {Functor φ} → φ α → Free φ α
  ≔ impure ○ map pure.
\end{program}

\begin{program}
instance Functor φ ⇒ Monad (Free φ)
  { map f fm ≔ cases fm
                { pure a ⇒ pure (f a)
                ; impure m ⇒ impure (map (map f) m) }
  ; return a ≔ pure a
  ; fm >>= k ≔ cases fm
                  { pure a ⇒ k a
                  ; impure m ⇒ impure (map (m' ⇒ m' >>= k) m) } }.
\end{program}

% -----------------------------------------------------------------------------
\subsubsection{Example: Free Mutable}

\begin{program}
type Free-Mutable (σ : Type) : Type → Type := Free (Mutable σ).

term get-Free (σ : Type) : Free-Mutable σ σ
  ≔ lift-Free get.

term set-Free (σ : Type) : Free-Mutable σ unit
  ≔ lift-Free set.

// TODO: did I define run-Mutable in monadic-effects?
term run-Free-Mutable
      (m : Free-Mutable σ α) (s : σ)
  : σ ✕ α
  ≔ cases m
      { pure a ⇒ (a, s)
      ; impure m' ⇒ let (m'', s') ≔ run-Mutable m' s in
                     run-Free-Mutable m'' s' }.
\end{program}

% -----------------------------------------------------------------------------
% -----------------------------------------------------------------------------
\subsection{Left Kan Extension}

\TODO{Intuition for freer monad using left Kan extension.}

\begin{program}
type LAN (γ : Type → Type) (α χ : Type) : Type
  ≔ (γ α) ✕ (χ → α) → LAN γ α.

term make-LAN (γ : Type → Type) (α χ : Type)
              (gx : γ α) (h : χ → α)
  : LAN γ α χ
  ≔ (gx, h).

instance Functor (LAN γ)
  { map f (make-LAN gx h) ≔ make-LAN gx (f ○ h) }.

term lift-LAN (γ : Type → Type) (α : Type) : γ α → LAN γ α
  ≔ (h gx ⇒ make-LAN gx h) (x ⇒ x).
\end{program}
